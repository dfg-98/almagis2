\chapter{Introducción}\label{chapter:introduction}


Un Sistema de Información Geográfica (GIS, por sus siglas en inglés) es
un conjunto integrado de software y hardware que permite la captura,
almacenamiento, procesamiento, análisis y visualización de datos
geográficamente referenciados\footnote{Georreferenciación. La
  georreferenciación es la técnica de posicionamiento espacial de una
  entidad en una localización geográfica única y bien definida en un
  sistema de coordenadas y datum específicos.}.  [\cite{uwmi-libraries-gis}]


La capacidad de GIS para vincular datos geoespaciales con información descriptiva abre
nuevas dimensiones en el análisis y la interpretación de datos en
numerosas disciplinas. Como se ha definido, un GIS no es solo un mapa
digital; es una herramienta compleja y poderosa para manejar y analizar
datos georreferenciados.


\section{Historia y Evolución del GIS}\label{historia-y-evolucion-del-gis}

El concepto de GIS se materializó por primera vez con el desarrollo del
Canadian Geographical Information System (CGIS) en 1962, liderado por
Roger Tomlinson\footnote{Roger Tomlinson. Geógrafo inglés residente en
  Canadá y principal artífice de los modernos GIS computarizados. Está
sistema fue una innovación significativa en el manejo de datos
geográficos para el Inventario de Tierras de Canadá (Canada Land
Inventory, CLI)}.
A lo largo de las décadas, el GIS ha evolucionado desde sistemas 
basados en mainframes a sistemas más sofisticados y accesibles en computadoras personales.[\cite{grindgis-gis}] 


Con la proliferación de Internet\footnote{Internet. Conjunto
  descentralizado de redes de comunicación interconectadas que utilizan
  la familia de protocolos TCP/IP.} en 1983 \parencite{andrews-who-invented-the-internet} y la posterior
emergencia de la World Wide Web\footnote{Web. La World Wide Web (WWW) o red informática mundial, es un sistema de distribución de documentos
  de hipertexto o hipermedia interconectados y accesibles a través de
  Internet. Es uno de los servicios que más éxito ha tenido en internet,
  hasta tal punto que es habitual la confusión entre ambos términos.} 
  
en 1990 \parencite{abbate-internet}, la distribución y el acceso a la información geográfica
experimentaron una transformación radical. Esta transición digital
facilitó el nacimiento de los Web GIS, que democratizan el acceso a
herramientas GIS a través de la web, permitiendo su uso desde cualquier
dispositivo con conexión a Internet y un navegador web.


\subsection{Objeto de Estudio: Web
GIS}\label{objeto-de-estudio-web-gis}

Esri\footnote{Esri. Esri (Environmental Systems Research Institute) es
  una empresa que actualmente desarrolla y comercializa software para
  Sistemas de Información Geográfica y es una de las compañías líderes a
  nivel mundial.} define el Web GIS como un sistema de información
distribuida que consta de al menos un servidor GIS y un cliente, siendo
este último un navegador web, una aplicación de escritorio o una
aplicación móvil. Esta tecnología emergió en Silicon Valley, California,
cuando Xerox desarrolló un simple visor de mapas web en 1993. Desde
entonces, los Web GIS han evolucionado hasta convertirse en plataformas
sofisticadas que ofrecen una amplia gama de funcionalidades.[\cite{fu-getting-to-know-web-gis}]

El enfoque de este estudio es examinar los Web GIS en profundidad. Se
explorará cómo estos sistemas, accesibles a través de diversas
plataformas y dispositivos, representan una evolución significativa
respecto a los GIS tradicionales, especialmente en términos de
accesibilidad, usabilidad y alcance.


\subsection{Motivación y
Justificación}\label{motivacion-y-justificacion}

El proyecto Cadic-UH en la Casa del Software, perteneciente a la
facultad de Matemática y Computación de la Universidad de la Habana, ha
orientado el desarrollo de AlmaGIS 2. Este Web GIS, diseñado como una
aplicación web adaptable para la gestión de información geográfica junto
al deseo personal por profundizar en el conocimiento de los sistemas de
información geográfica y el análisis espacial constituyen la motivación
principal de este trabajo.


\subsection{Requerimientos Funcionales y de
Entorno}\label{requerimientos-funcionales-y-de-entorno}

El desarrollo de AlmaGIS 2 se basará en una serie de requerimientos
funcionales y de entorno específicos, detallados a continuación, para
garantizar su eficacia y adaptabilidad:

\begin{enumerate}
\def\labelenumi{\arabic{enumi}.}
\item
  \textbf{Visor Cartográfico Generalizado y Adaptable}: Implementación
  de un visor para cartografías que sea capaz de adaptarse a distintos
  dominios y contextos de uso.
\item
  \textbf{Funcionalidades Interactivas Estándar}: Inclusión de
  funcionalidades de interacción comunes como zoom, navegación
  panorámica (pan), y gestión de capas (activar/desactivar).
\item
  \textbf{Búsqueda y Selección de Elementos del Dominio}: Herramientas
  avanzadas para la búsqueda de elementos específicos dentro del dominio
  cartográfico. Esta función permitirá seleccionar y destacar elementos
  individuales dentro de un conjunto de resultados.
\item
  \textbf{Áreas de Interés y Consultas Asociadas}: Capacidad para
  seleccionar áreas de interés específicas y realizar consultas de datos
  relacionadas con estas áreas.
\item
  \textbf{Consulta de Datos Alfanuméricos}: Función para visualizar
  datos alfanuméricos asociados a elementos seleccionados en la
  cartografía, basados en el área de interés definida.
\item
  \textbf{Análisis Estadístico de Grupos y Variables Nominales}:
  Herramientas para realizar análisis estadísticos agrupados y por
  variables nominales, permitiendo una comprensión más profunda de los
  conjuntos de datos.
\item
  \textbf{Sistema de Autenticación y Gestión de Usuarios}:
  Implementación de un sistema robusto de autenticación de usuarios con
  capacidades administrativas para gestionar permisos y restricciones de
  acceso.
\item
  \textbf{Administración Avanzada de Datos}: Herramientas para una
  gestión eficiente de los datos almacenados, incluyendo la visibilidad
  y opacidad de las capas, así como la administración de usuarios y sus
  restricciones.
\item
  \textbf{Herramientas de Medición y Edición Cartográfica}: Inclusión de
  herramientas para medir distancias y áreas, así como para la edición
  de mapas que permitan resaltar zonas o elementos específicos.
\item
  \textbf{Funcionalidades de Búsqueda Diversificadas}: Capacidades de
  búsqueda avanzadas en el mapa para facilitar la localización y
  análisis de datos específicos.
\item
  \textbf{Generación y Extensibilidad de Reportes}: Creación de reportes
  de forma extensible, permitiendo la adaptación a diferentes
  necesidades de información.
\item
  \textbf{Exportación de Datos en Formatos Vectoriales}: Posibilidad de
  descargar geometrías resultantes de las búsquedas en formatos
  vectoriales como WKT y GeoJSON.
\item
  \textbf{Uso de Tecnologías de Código Abierto Actuales}: Aplicación de
  tecnologías de código abierto modernas y ampliamente utilizadas para
  asegurar la accesibilidad y la actualización continua del sistema.
\item
  \textbf{Compatibilidad Multiplataforma}: El software debe ser
  funcional y eficiente tanto en sistemas operativos Windows como Linux,
  garantizando así una amplia accesibilidad.
\end{enumerate}

Estos requerimientos son fundamentales para el desarrollo de un sistema
Web GIS que sea a la vez poderoso, flexible y accesible para una amplia
gama de usuarios y aplicaciones. La implementación exitosa de estas
características permitirá que AlmaGIS 2 se destaque como una herramienta
innovadora en el campo de los sistemas de información geográfica.

Dos componentes fundamentales en un Web GIS son:

\begin{enumerate}
\def\labelenumi{\arabic{enumi}.}
\item
  Visor de mapas: Aplicación web pensada para la visualización y la
  consulta de información geográfica haciendo uso de los estándares de
  la OGC \footnote{OGC. El Open Geospatial Consortium (OGC) fue creado
    en 1994. Es una organización internacional sin fines de lucro
    comprometida con la creación de estándares abiertos e interoperables
    dentro de los Sistemas de Información Geográfica y la World Wide
    Web.}. \parencite{andorra-visor-de-mapas}
\item
  Servidor de mapas (en inglés IMS: Internet Map Server): Provee
  cartografía a través de la red tanto en modo vectorial como con
  imágenes. La especificación estándar para estos servidores es la OGC
  WMS (Open Geospatial Consortium Web Map Service). \parencite{panorama-sig-libre-servidores}
\end{enumerate}

Es posible la realización del proyecto gracias a la gran experiencia que
posee La Casa del Software en la realización de GIS y aplicaciones Web
GIS. Se utilizarán herramientas de código abierto\footnote{Código
  abierto. Es el software distribuido y desarrollado libremente.
  Cualquier persona o entidad puede utilizarlo y modificar su código
  fuente.} como GeoServer, Net Core y Angular en sus versiones más
recientes.


\subsection{Problema}\label{problema}

El problema central que aborda esta investigación se articula en torno a
una interrogante fundamental: ¿Cómo diseñar e implementar, utilizando
buenas prácticas de programación, un sistema Web GIS que satisfaga
integralmente todos los requerimientos funcionales y de entorno
detallados previamente? Este desafío implica no solo la creación técnica
de un sistema de información geográfica accesible a través de la web,
sino también la garantía de que dicho sistema sea robusto, eficiente, y
fácilmente adaptable a diversos contextos y necesidades de los usuarios.

Este desafío se desglosa en varios aspectos clave:

\begin{enumerate}
\def\labelenumi{\arabic{enumi}.}
\item
  \textbf{Integración de Funcionalidades Avanzadas y Diversificadas}:
  Desarrollar un sistema que no solo ofrezca las funcionalidades
  estándar de un GIS, sino que también incorpore herramientas avanzadas
  de análisis y visualización de datos, adaptabilidad a distintos
  dominios y capacidades de interacción intuitiva para el usuario.
\item
  \textbf{Arquitectura Sostenible y Escalable}: Establecer una
  arquitectura de software que no solo atienda las necesidades actuales,
  sino que también sea capaz de adaptarse y escalar para futuras
  expansiones y mejoras, manteniendo la estabilidad y eficiencia del
  sistema.
\item
  \textbf{Interfaz de Usuario Intuitiva y Accesible}: Diseñar una
  interfaz que sea al mismo tiempo poderosa en capacidades técnicas y
  accesible para usuarios con variados niveles de experiencia en GIS,
  desde especialistas hasta usuarios ocasionales.
\item
  \textbf{Cumplimiento de Estándares de Desarrollo de Software}:
  Asegurar que todo el proceso de desarrollo se adhiera a las mejores
  prácticas y estándares de la industria del software, incluyendo la
  codificación, el testing, y la documentación, para garantizar un
  producto final de alta calidad.
\item
  \textbf{Integración de Tecnologías de Código Abierto}: Elegir y
  utilizar eficazmente tecnologías de código abierto, las cuales deben
  ser actuales, robustas y ampliamente soportadas, para asegurar la
  sostenibilidad y la interoperabilidad del sistema.
\item
  \textbf{Compatibilidad y Funcionamiento Multiplataforma}: Garantizar
  que el sistema sea completamente funcional y consistente en distintos
  sistemas operativos y dispositivos, lo que implica un enfoque
  meticuloso en el diseño de software multiplataforma.
\item
  \textbf{Gestión de Datos y Seguridad}: Implementar un sistema efectivo
  para la gestión de datos, que incluya no solo la manipulación
  eficiente de grandes volúmenes de datos geoespaciales sino también la
  seguridad y privacidad de dichos datos.
\end{enumerate}

En resumen, el problema planteado no se limita a la mera implementación
técnica de un Web GIS, sino que abarca la creación de un sistema
integral, que sea tanto innovador en sus capacidades técnicas como
accesible y útil para una amplia gama de usuarios. El objetivo es
superar los retos actuales en el campo de los sistemas de información
geográfica web y establecer un nuevo estándar en la funcionalidad,
usabilidad y adaptabilidad de estos sistemas.


\subsection{Objetivos}\label{objetivos}


\subsubsection{Objetivos Generales}\label{objetivos-generales}

\begin{itemize}
\item
  Diseñar e implementar un Web GIS que satisfaga todos los
  requerimientos funcionales y de entorno.
\item
  Asegurar que el Web GIS sea extensible y esté construido siguiendo las
  mejores prácticas de programación.
\item
  Incorporar las tecnologías más avanzadas y relevantes en el desarrollo
  del Web GIS.
\end{itemize}


\paragraph{Objetivos Específicos}\label{objetivos-especuxedficos}

\begin{itemize}
\item
  Analizar el estado actual de los Web GIS con funcionalidades similares
  a las deseadas.
\item
  Desarrollar AlmaGIS 2, una nueva aplicación que utilice tecnologías
  modernas y de código abierto.
\item
  Diseñar e implementar una arquitectura extensible que permita una
  gestión eficiente de usuarios y datos.
\item
  Implementar herramientas específicas que satisfagan los requerimientos
  funcionales del proyecto.
\end{itemize}


\subsubsection{Estructura del Trabajo}\label{estructura-del-trabajo}

Este trabajo se divide en los siguientes capítulos:

\begin{itemize}
\item
  \textbf{Capítulo 1}: Introducción
\item
  \textbf{Capítulo 2}: Estado del Arte - Análisis de los Web GIS
  existentes.
\item
  \textbf{Capítulo 3}: Solución Teórico-Conceptual-Computacional -
  Arquitectura de software y patrones de diseño utilizados.
\item
  \textbf{Capítulo 4}: Detalles de Implementación - Tecnologías y
  herramientas utilizadas.
\item
  \textbf{Capítulo 5}: Pruebas de Funcionalidad - Demostración del
  funcionamiento del software.
\item
  \textbf{Capítulo 6}: Conclusiones - Reflexiones finales y posibles
  direcciones futuras.
\end{itemize}

Los anexos proporcionarán vistas adicionales de la aplicación y detalles
técnicos que no se incluyen en los capítulos principales.



